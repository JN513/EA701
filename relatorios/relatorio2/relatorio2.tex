\documentclass{article}
\usepackage[utf8]{inputenc}
\usepackage[T1]{fontenc}
\usepackage[brazil]{babel}
\usepackage{geometry}
\usepackage{graphicx}
\usepackage{float}
\usepackage{array}
\usepackage{booktabs}
\usepackage{amsmath}
\usepackage{hyperref}

\title{Relatório Experimental - GPIOs e PWM na BitDogLab}
\author{Ana Beatriz Barbosa Yoshida - RA: 245609 \\ Julio Nunes Avelar - RA: 241163}
\date{3 de Setembro de 2025}

\begin{document}

\maketitle

\section{Objetivos}
\begin{itemize}
    \item Gerar um sinal PWM ajustável em frequência e duty cycle (GPIO0).
    \item Medir esse sinal através de um pino de probe (GPIO1) e exibir os valores no display OLED.
    \item Avaliar a integridade do sinal em diferentes condições de conexão, incluindo degradação intencional com protoboards e cabos.
    \item Validar os resultados com o auxílio de um osciloscópio.
\end{itemize}

\section{Materiais Utilizados}
\begin{itemize}
    \item BitDogLab
    \item Protoboard
    \item Jumpers
    \item Osciloscópio
    \item 1 LED vermelho
    \item 1 resistor de 220 ohm
    \item Outros
\end{itemize}

\section{Procedimentos}

\subsection{Configuração do Gerador PWM}
\begin{itemize}
    \item Programe a GPIO0 como saída PWM.
    \item Configure a frequência e o duty cycle como variáveis ajustáveis:
    \begin{itemize}
        \item Joystick eixo X: variação da frequência.
        \item Joystick eixo Y: variação do duty cycle.
        \item Botões A e B: ajuste fino (incrementar/decrementar).
    \end{itemize}
    \item Sugestão: Teste o sinal conectando um LED na GPIO0 com resistor limitador (opcional, apenas para visualização qualitativa).
\end{itemize}
Explique como foi configurada a GPIO0 para gerar o sinal PWM e como os controles (joystick e botões) foram utilizados. \\
Inclua o link do Github do seu código. 

\subsection{Implementação do Probe Digital}
\begin{itemize}
    \item Programe a GPIO2 como entrada digital.
    \item Faça a conexão física GPIO0 $\rightarrow$ GPIO1 com um jumper.
    \item Desenvolva uma rotina que:
    \begin{itemize}
        \item Conte pulsos de subida e descida para calcular a frequência.
        \item Meça o tempo em nível alto e baixo para calcular o duty cycle.
        \item Exiba os valores medidos no OLED em tempo real.
    \end{itemize}
\end{itemize}
Explique como foi feita a conexão física (GPIO0 $\rightarrow$ GPIO1) e como o programa realizou a medição de frequência e duty cycle. \\
Inclua o link do Github do seu código. 

\subsection{Testes Diretos}
Ajuste valores distintos de frequência e duty cycle (ex.: 100 Hz, 1 kHz, 10 kHz; duty 25\%, 50\%, 75\%). \\

\begin{itemize}
    \item Conecte o canal 1 do osciloscópio na GPIO0 (sinal gerado).
    \item Conecte o canal 2 na GPIO1 (sinal após degradação).
    \item Compare:
    \begin{itemize}
        \item Forma de onda (bordas, ruído, integridade).
        \item Valores medidos internamente (OLED) $\times$ osciloscópio.
    \end{itemize}
\end{itemize}

Anote os valores programados e os valores medidos no OLED.


\begin{table}[H]
    \centering
    \label{tab:dados_coletados}
    \begin{tabular}{lccccc}
        \toprule
        \textbf{Freq. Programada} & \textbf{Duty Programado} & \textbf{Freq. Medida (OLED)} & \textbf{Duty Medido (OLED)} & \textbf{Osciloscópio} & \textbf{Observações} \\
        \midrule
        & & & & & \\
        & & & & & \\
        \bottomrule
    \end{tabular}
    \caption{Dados coletados durante os testes.}
\end{table}

\subsection{Testes com Degradação do Sinal}
Monte o caminho do sinal passando por protoboards, fios longos e conectores extras. \\

Observe no OLED e no osciloscópio:
\begin{itemize}
    \item Mudanças de frequência/duty cycle medidos.
    \item Distorção de borda ou presença de ruídos.
\end{itemize}

Registre fotos e um vídeo curto da montagem e dos resultados. \\
Descreva a montagem com fios longos, protoboards e conectores. Inclua fotos da montagem. \\
Apresente nova tabela com os resultados obtidos.

\section{Resultados}
Apresente as diferenças observadas entre os valores programados, medidos pelo OLED e confirmados no osciloscópio. \\
Insira gráficos ou capturas de tela do osciloscópio, se disponíveis.

\section{Discussão}
Analise os resultados obtidos:
\begin{itemize}
    \item Houve diferença significativa entre o valor teórico e o medido?
    \item Como a degradação do sinal afetou as medidas?
    \item O que o osciloscópio mostrou de diferente em relação ao OLED?
    \item Quais fatores elétricos explicam essas diferenças (ruído, capacitância da protoboard, resistência dos cabos, etc.)?
\end{itemize}

\section{Conclusão}
Resuma em 1 ou 2 parágrafos o que foi aprendido com a atividade em termos de:
\begin{itemize}
    \item Uso de GPIOs e PWM.
    \item Medição de sinais digitais.
    \item Influência da qualidade das conexões na integridade do sinal.
\end{itemize}

\end{document}
