\documentclass{article}
\usepackage[utf8]{inputenc}
\usepackage[T1]{fontenc}
\usepackage[brazil]{babel}
\usepackage{geometry}
\usepackage{graphicx}
\usepackage{float}
\usepackage{array}
\usepackage{booktabs}
\usepackage{amsmath}
\usepackage{hyperref}
\usepackage{xcolor}
\usepackage{makecell}
\usepackage{subcaption} % no preâmbulo

\title{Relatório Experimental - Persistência em Memória (RAM vs Flash)}
\author{Ana Beatriz Barbosa Yoshida - RA: 245609 \\ Julio Nunes Avelar - RA: 241163}
\date{17 de Setembro de 2025}

\begin{document}

\maketitle

\section{Objetivos}

Compreender a diferença entre memória volátil (RAM) e memória não volátil (Flash) no RP2040, experimentando formas de salvar e recuperar informações no microcontrolador.

Aplicar isso em um exemplo prático: guardar o “estado” de um LED RGB e restaurá-lo após reiniciar a placa.

\section{Tarefa}

\begin{enumerate}
    \item Compare o uso de RAM (variáveis normais) versus Flash (persistência após reset).
    \item Meça o tempo de gravação em cada caso (em ms)
    \item Discuta
    \begin{enumerate}
        \item Quando é necessário gravar na Flash?
        \item Quais riscos existem (ex.: desgaste, energia no meio da escrita)?
        \item Que boas práticas adotar (salvar só quando necessário, agrupar dados em estrutura única, etc.).
    \end{enumerate}
\end{enumerate}

\end{document}
